\documentclass{article}
\usepackage[T1]{fontenc} % Soporte para español
\usepackage[utf8]{inputenc} % Codificación UTF-8
\usepackage{graphicx} % Para incluir imágenes
\usepackage{background} % Para configurar el fondo
\usepackage{geometry} % Para ajustar la geometría de las páginas

% Configuración del fondo para las páginas regulares
\backgroundsetup{
scale=1,
color=black,
opacity=0.1,
angle=0,
contents={%
  \includegraphics[width=\paperwidth,height=\paperheight]{fondo_pagina.png}
  }
}

\begin{document}

% Página de título con su propio fondo
{
\backgroundsetup{
contents={\includegraphics[width=\paperwidth,height=\paperheight]{fondo_portada.jpg}},
opacity=1,
scale=1
}
\begin{titlepage}
\newgeometry{left=3cm,right=3cm,top=2cm,bottom=2cm} % Margenes para la portada
\centering
\vspace*{5cm} % Espacio vertical antes del título

{\Huge\bfseries\textbf{Universidad Abierta y a Distancia de México}}\\[1cm]
{\Large\bfseries\textbf{División de Ciencias Exactas, Ingeniería y Tecnología}}\\[1cm]
{\Large\bfseries\textbf{Programa educativo: Matemáticas}}\\[0.5cm]
{\Large\bfseries\textbf{Semestre }}\\[0.5cm]
{\Large\bfseries\textbf{Unidad : }}\\[0.5cm]
{\Large\bfseries\textbf{Actividad: }}\\[2cm]
{\Large\bfseries\textbf{Nombre del estudiante: Iván Fernández Guerrero}}\\[0.5cm]
{\Large\bfseries\textbf{Matrícula: ES231104533}}\\[0.5cm]
{\Large\bfseries\textbf{Grupo: }}\\[0.5cm]
{\Large\bfseries\textbf{Figura Académica: }}\\[2cm]
{\Large\bfseries\textbf{Ciudad de México, a – de Mes de 2024}}\\[2cm]

\vfill
\end{titlepage}
\restoregeometry % Restaura la geometría de la página
} % Fin del grupo de la portada

% Contenido del documento
\section{Introducción}
En esta sección, se presenta una breve descripción del tema, los objetivos del trabajo y el contexto en el cual se desarrollará.

\newpage
\section{Desarrollo}
Esta sección contiene el desarrollo principal del tema, incluyendo análisis, discusión y resultados obtenidos durante la investigación.


\newpage
\section{Conclusión}
En esta sección se resumen los puntos principales del trabajo, se destacan los hallazgos más relevantes y se proponen posibles líneas de investigación futura.

\newpage
\section{Referencias}
Aquí se listan todas las fuentes y referencias bibliográficas utilizadas durante la elaboración del trabajo.

\begin{thebibliography}{9}
    \bibitem{referencia1} Autor, Título del libro, Editorial, Año.
    \bibitem{referencia2} Autor, "Título del artículo", Nombre de la revista, Volumen (Año), Páginas.
    \bibitem{referencia3} Autor, "Título del artículo en línea", Sitio web, Fecha de consulta.
\end{thebibliography}

\newpage

\end{document}
